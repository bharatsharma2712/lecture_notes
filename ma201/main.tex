\documentclass[oneside]{book}
\usepackage[a4paper, margin=3cm]{geometry}
\usepackage{import}
\usepackage{xcolor}
\usepackage{pdfpages}
\usepackage{transparent}
\usepackage{subcaption}
\usepackage{float}
\usepackage{amsmath}
\usepackage{amssymb}
\usepackage{hyperref}
\usepackage{babel}
\usepackage{cleveref}

\newcommand{\incfig}[1]{%
    \import{./figures/}{#1.pdf_tex}
}
\newcommand{\incfigsc}[2][1]{%
    \def\svgwidth{#1\columnwidth}
    \import{./figures/}{#2.pdf_tex}
}

\newcommand{\halfsubfig}[2][1]{
	\begin{subfigure}{0.45\linewidth}
		\centering
		\incfig{#1}
		\caption{#2}
	\end{subfigure}
}
\hypersetup{
    colorlinks,
    linkcolor={blue!80!black},
    citecolor={blue!50!black},
    urlcolor={blue!80!black}
}
\newtheorem{theorem}{Theorem}[chapter]

\title{MA201 - Differential Equations}
\author{Indian Institute of Technology Ropar}
\begin{document}
\maketitle
{
	\hypersetup{
		hidelinks
	}
	\tableofcontents
}
\renewcommand{\arraystretch}{1.5}%Sets the vertical margin

\chapter{Linear Differential Equations}
A general \(n^{th}\) order differential equation can be written as
\[
	F(x, y, \frac{dy}{dx}, \cdots, \frac{d^ny}{dx^n}) = 0
\]
An $n^{th}$ order ODE is \textit{linear}, if it can be written in the form
\[
	a_0(x)\frac{d^ny}{dx^n} + a_1(x)\frac{d^{n-1}y}{dx^{n-1}} + \cdots + a_n(x) y = g(x)
\]
where the functions \(a_i(x)\) are called the coefficient functions.
\\\\
\noindent A \textit{non-linear} ODE is an ODE which is NOT linear.

\section{Solution of an ODE}
A solution of the $n^{th}$ order ODE
\[
	F(x, y, \frac{dy}{dx}, \cdots, \frac{d^ny}{dx^n}) = 0
\]
on an interval \(I\subseteq \mathbb{R}\) is a function \(y = \phi(x)\) which is defined on $I$, which is at least $n$ times differentiable on $I$, and which satisfies the equation.

\section{First Order Ordinary Differential Equations}
Consider the first-order ODE of the form
\begin{equation}
	\frac{dy}{dx} = f(x, y)
	\label{first_order_general_form_1}
\end{equation}
The \cref{first_order_general_form_1} can always be written as
\begin{equation}
	M(x, y)dx + N(x, y)dy = 0
	\label{first_order_general_form_2}
\end{equation}
\subsection{Solving First Order ODEs}
We assume that the ODE of the form \cref{first_order_general_form_1} or \cref{first_order_general_form_2} has a solution.
\begin{enumerate}
	\item \textbf{Separable Equations}\\
	      If the \cref{first_order_general_form_1} or \cref{first_order_general_form_2} can be written in the form
	      \[
		      \frac{dy}{dx} = g(x)\cdot h(y)
	      \]
	      or
	      \[
		      f_1(x)\phi_1(y)dx + f_2(x)\phi_2(y)dy = 0
	      \]
	      then the DE is called a separable equation and can be solved by separating variables and integrating.
	\item \textbf{Homogeneous Equations}\\
	      If the \cref{first_order_general_form_2} can be written in the form
	      \[
		      \frac{dy}{dx} = f(\frac{x}{y})
	      \]
	      then the DE is called a homogeneous equation and can be solved by substituting \(y = vx\) and \(\displaystyle \frac{dy}{dx} = v + x\frac{dv}{dx}\). After substitution, the equation changes to separable equation.
	\item \textbf{Exact Equations}\\
	      If the \cref{first_order_general_form_2} can be written in the form
	      \[
		      dF(x, y) = 0
	      \]
	      without multiplying by any factor,
	      then the DE is called a homogeneous equation and its general solution solution is
	      \[
		      F(x, y) = c
	      \]
	      where $c$ is an arbitrary constant.
\end{enumerate}
\textbf{Note:} Total differential $:= dF$ and it is defined by the formula
\[
	dF(x, y) = \frac{\partial F(x, y)}{\partial x}dx + \frac{\partial F(x, y)}{\partial y}dy
\]
\subsection{Exact Differential Equations}
How to check if a DE of form \cref{first_order_general_form_2} is exact or not.
\begin{theorem}
	Consider the differential equation
	\begin{equation}
		M(x, y)dx + N(x, y)dy = 0
		\label{exact_de_start_equation}
	\end{equation}
	where $M(x, y)$ and $N(x, y)$ have continuous first partial derivatives at all points $(x, y)$ in rectangular domain $D$. Then, the differential equation \ref{exact_de_start_equation} is exact \textbf{iff}
	\[
		\frac{\partial M}{\partial y}(x, y) = \frac{\partial N}{\partial x}(x, y)
	\]
\end{theorem}

\subsection{Integrating Factors}
\label{integrating_factor_subsection}
Sometimes an equation \textbf{not exact} but can be \textbf{made exact} by multiplying it by some function of $x$ and $y$. The function which when multiplied, makes the equation exact is called \textit{integrating factor}.
\renewcommand{\labelenumi}{\textbf{\arabic{enumi}:}}
\begin{enumerate}
	\item If
	      \(
	      \displaystyle\frac{1}{N}\left[\frac{\partial M}{\partial y} - \frac{\partial N}{\partial x}\right]
	      \)
	      be a function of $x$ only
	      \[
		      \frac{1}{N}\left[\frac{\partial M}{\partial y} - \frac{\partial N}{\partial x}\right] = g(x)
	      \]
	      then \(\displaystyle e^{\int g(x) dx}\) is an IF of the equation.
	      \label{rule_1_for_integrating_factor}
	\item If
	      \(
	      \displaystyle\frac{1}{M}\left[\frac{\partial N}{\partial x} - \frac{\partial M}{\partial y}\right]
	      \)
	      be a function of $y$ only
	      \[
		      \frac{1}{M}\left[\frac{\partial N}{\partial x} - \frac{\partial M}{\partial y}\right] = h(y)
	      \]
	      then \(\displaystyle e^{\int h(y) dy}\) is an IF of the equation.
	\item If
	      \(
	      \displaystyle Mx + Ny \neq 0
	      \)
	      and the equation is homogeneous,
	      then \(\displaystyle \frac{1}{Mx + Ny}\) is an IF of the equation.
	\item If
	      \(
	      \displaystyle Mx - Ny \neq 0
	      \)
	      and the equation can be written as
	      \[
		      \left\{f_1(xy)\right\}ydx + \left\{f_2(xy)\right\}xdy = 0
	      \]
	      then \(\displaystyle \frac{1}{Mx - Ny}\) is an IF of the equation.
\end{enumerate}

\subsection{First Order Linear Differential Equation}
The first order ODE is linear in the dependent varialbe $y$ and independent variable $x$ if it can be written in the form
\begin{equation}
	\frac{dy}{dx} + P(x)y = Q(x)
	\label{first_order_linear_de_general}
\end{equation}
where $P$ and $Q$ are function of $x$ only.

\subsubsection{Solution}
\begin{itemize}
	\item If $P(x) = 0$, then the \cref{first_order_linear_de_general} degenerates into a simple separable equation.
	\item If $P(x) != 0$, then the \cref{first_order_linear_de_general} is \textbf{not} exact. By rule \ref{rule_1_for_integrating_factor} of \cref{integrating_factor_subsection}, integrating factor is \(\displaystyle e^{\int P(x)dx}\). Therefore, the solution is
	      \[
		      y(x) = \frac{\int Q(x)e^{\int P(x)dx}dx}{e^{\int P(x)dx}}
	      \]
	      where $c$ is a constant.
\end{itemize}

\subsection{Bernoulli Equations}
An equation of the form
\begin{equation}
	\frac{dy}{dx} + P(x) y  = Q(x) y^n
	\label{bernoulli_equation_general}
\end{equation}
is called a Bernoulli DE, where $P$ and $Q$ are functions of $X$ alone.
\\\\
\noindent\textbf{Note:} When $n\in\{0, 1\}$, then \cref{bernoulli_equation_general} is a linear DE.

\begin{theorem}
	Suppose \(n\notin\{0, 1\}\), then the transformation \(v = y^{1-n}\) reduces the \cref{bernoulli_equation_general} to a linear DE in \(v\).
\end{theorem}

\section{Initial Value Problems}
Consider the first order DE of the form
\[
	\begin{cases}
		\frac{dy}{dx} = f(x, y) \\
		y(x_0) = y_0
	\end{cases}
\]
where \(f\) is a real-valued function defined in some domain \(D\) in \(xy\)-plane.
\[
	D = \{(x, y): a\leq x \leq b, c\leq y \leq d\}
\]
where \(a, b, c\) and \(d\) re real finite constants.\\

\noindent This type of DE is called \textit{initial value problem}(IVP). and \(y(x_0) = y_0\) is called an \textit{initial condition}.
A IVP may have zero, one, finite or infinite solutions.

\subsection{Peano Existence Theorem}
\noindent\textit{Let \(D\) be a rectangular domain that contains point \(x_0, y_0\) in its interior. If \(f(x, y)\) is a \textbf{continuous} function in the domain \(D\), then there exists a solution to the initial value problem on some interval \(I = (x_0 - h, x_0 + h)\) where \(h>0\), sufficiently small.}

\subsection{Picard Existence and Uniqueness Theorem}
\noindent\textit{Let \(D\) be a rectangular domain that contains point \(x_0, y_0\) in its interior. If \(f(x, y)\) and \(\frac{\partial f}{\partial y}(x, y)\) are \textbf{continuous} function in the domain \(D\), then there exists some interval \(I = (x_0 - h, x_0 + h)\), \(h>0\) contained in \([a, b]\) and a \textbf{unique} function \(\phi(x)\) defined on \(I\) that is a solution of the initial value problem.}
\\\\
\noindent\textbf{Note:} If \(f(x, y)\) does not satisfy the hypotheses of \textit{Picard Existence and Uniqueness Theorem}, then we \textbf{cannot} conclude about the solution.

\subsection{Lipschitz Condition}
\noindent\textit{The function \(f: D\to\mathbb{R}\) is said to be \textsc{Lipschitz} w.r.t. \(y\) in \(D\) if \(\exists\ k>0\) such that}
\[
	|f(x, y_1) - f(x, y_2)| \leq k|y_1 - y_2|
\]
\textit{for all \((x, y_1), (x, y_2) \in D\).}
\\\\
\noindent \(k\) is called \textit{Lipschitz constant}.

\subsection{Picard Existence and Uniqueness Theorem - 2}
\textit{Let \(D\) be a rectangular domain that contains point \((x_0, y_0)\) in its interior. Let \(f\) satisfies the following two conditions:}
\begin{enumerate}
	\item \textit{\(f(x, y)\) is continuous in \(D\)}
	\item \textit{\(f(x, y)\) satisfies a Lipschitz condition w.r.t. in \(y\) in \(D\)}
\end{enumerate}
\textit{Consider the rectangle \(R: |x - x_0| \leq \alpha, |y- y_0|\leq \beta\) in \(D\). Then, there exists a \textbf{unique} solution to the IVP in the interval \(I = [x_0 - h, x_0 + h]\), where \(h = min\{\alpha, \frac{\beta}{M}\}, M = \max\limits_{(x, y)\in\mathbb{R}}|f(x, y)|\)}.


\bibliographystyle{unsrt}
\bibliography{main}
\end{document}