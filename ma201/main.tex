\documentclass[oneside]{book}
\usepackage[a4paper, margin=3cm]{geometry}
\usepackage{import}
\usepackage{xcolor}
\usepackage{pdfpages}
\usepackage{transparent}
\usepackage{subcaption}
\usepackage{float}
\usepackage{amsmath}
\usepackage{amssymb}
\usepackage{hyperref}
\usepackage{babel}
\usepackage{cleveref}

\newcommand{\incfig}[1]{%
    \import{./figures/}{#1.pdf_tex}
}
\newcommand{\incfigsc}[2][1]{%
    \def\svgwidth{#1\columnwidth}
    \import{./figures/}{#2.pdf_tex}
}

\newcommand{\halfsubfig}[2][1]{
	\begin{subfigure}{0.45\linewidth}
		\centering
		\incfig{#1}
		\caption{#2}
	\end{subfigure}
}
\hypersetup{
    colorlinks,
    linkcolor={blue!80!black},
    citecolor={blue!50!black},
    urlcolor={blue!80!black}
}
\newtheorem{theorem}{Theorem}[chapter]

\title{MA201 - Differential Equations}
\author{Indian Institute of Technology Ropar}
\begin{document}
\maketitle
{
	\hypersetup{
		hidelinks
	}
	\tableofcontents
}
\renewcommand{\arraystretch}{1.5}%Sets the vertical margin

\chapter{Linear Differential Equations}
A general \(n^th\) order differential equation can be written as
\[
	F(x, y, \frac{dy}{dx}, \cdots, \frac{d^ny}{dx^n}) = 0
\]
An $n^{th}$ order ODE is \textit{linear}, if it can be written in the form
\[
	a_0(x)\frac{d^ny}{dx^n} + a_1(x)\frac{d^{n-1}y}{dx^{n-1}} + \cdots + a_n(x) y = g(x)
\]
where the functions \(a_i(x)\) are called the coefficient functions.
\\\\
\noindent A \textit{non-linear} ODE is an ODE which is NOT linear.

\section{Solution of an ODE}
A solution of the $n^{th}$ order ODE
\[
	F(x, y, \frac{dy}{dx}, \cdots, \frac{d^ny}{dx^n}) = 0
\]
on an interval \(I\subseteq \mathbb{R}\) is a function \(y = \phi(x)\) which is defined on $I$, which is at least $n$ times differentiable on $I$, and which satisfies the equation.

\section{First Order Ordinary Differential Equations}
Consider the first-order ODE of the form
\begin{equation}
	\frac{dy}{dx} = f(x, y)
	\label{first_order_general_form_1}
\end{equation}
The \cref{first_order_general_form_1} can always be written as
\begin{equation}
	M(x, y)dx + N(x, y)dy = 0
	\label{first_order_general_form_2}
\end{equation}
\subsection{Solving First Order ODEs}
We assume that the ODE of the form \cref{first_order_general_form_1} or \cref{first_order_general_form_2} has a solution.
\begin{enumerate}
	\item \textbf{Separable Equations}\\
	      If the \cref{first_order_general_form_1} or \cref{first_order_general_form_2} can be written in the form
	      \[
		      \frac{dy}{dx} = g(x)\cdot h(y)
	      \]
	      or
	      \[
		      f_1(x)\phi_1(y)dx + f_2(x)\phi_2(y)dy = 0
	      \]
	      then the DE is called a separable equation and can be solved by separating variables and integrating.
	\item \textbf{Homogeneous Equations}\\
	      If the \cref{first_order_general_form_2} can be written in the form
	      \[
		      \frac{dy}{dx} = f(\frac{x}{y})
	      \]
	      then the DE is called a homogeneous equation and can be solved by substituting \(y = vx\) and \(\displaystyle \frac{dy}{dx} = v + x\frac{dv}{dx}\). After substitution, the equation changes to separable equation.
	\item \textbf{Exact Equations}\\
	      If the \cref{first_order_general_form_2} can be written in the form
	      \[
		      dF(x, y) = 0
	      \]
	      without multiplying by any factor,
	      then the DE is called a homogeneous equation and its general solution solution is
	      \[
		      F(x, y) = c
	      \]
	      where $c$ is an arbitrary constant.
\end{enumerate}
\textbf{Note:} Total differential $:= dF$ and it is defined by the formula
\[
	dF(x, y) = \frac{\partial F(x, y)}{\partial x}dx + \frac{\partial F(x, y)}{\partial y}dy
\]
\subsection{Exact Differential Equations}
How to check if a DE of form \cref{first_order_general_form_2} is exact or not.
\begin{theorem}
	Consider the differential equation
	\begin{equation}
		M(x, y)dx + N(x, y)dy = 0
		\label{exact_de_start_equation}
	\end{equation}
	where $M(x, y)$ and $N(x, y)$ have continuous first partial derivatives at all points $(x, y)$ in rectangular domain $D$. Then, the differential equation \ref{exact_de_start_equation} is exact \textbf{iff}
	\[
		\frac{\partial M}{\partial y}(x, y) = \frac{\partial N}{\partial x}(x, y)
	\]
\end{theorem}

\subsection{Integrating Factors}
Sometimes an equation \textbf{not exact} but can be \textbf{made exact} by multiplying it by some function of $x$ and $y$. The function which when multiplied, makes the equation exact is called \textit{integrating factor}.
\renewcommand{\labelenumi}{\textbf{\arabic{enumi}:}}
\begin{enumerate}
	\item If
	      \(
	      \displaystyle\frac{1}{N}\left[\frac{\partial M}{\partial y} - \frac{\partial N}{\partial x}\right]
	      \)
	      be a function of $x$ only
	      \[
		      \frac{1}{N}\left[\frac{\partial M}{\partial y} - \frac{\partial N}{\partial x}\right] = g(x)
	      \]
	      then \(\displaystyle e^{\int g(x) dx}\) is an IF of the equation.
	\item If
	      \(
	      \displaystyle\frac{1}{M}\left[\frac{\partial N}{\partial x} - \frac{\partial M}{\partial y}\right]
	      \)
	      be a function of $y$ only
	      \[
		      \frac{1}{M}\left[\frac{\partial N}{\partial x} - \frac{\partial M}{\partial y}\right] = h(y)
	      \]
	      then \(\displaystyle e^{\int h(y) dy}\) is an IF of the equation.
	\item If
	      \(
	      \displaystyle Mx + Ny \neq 0
	      \)
	      and the equation is homogeneous,
	      then \(\displaystyle \frac{1}{Mx + Ny}\) is an IF of the equation.
	\item If
	      \(
	      \displaystyle Mx - Ny \neq 0
	      \)
	      and the equation can be written as
	      \[
		      \left\{f_1(xy)\right\}ydx + \left\{f_2(xy)\right\}xdy = 0
	      \]
	      then \(\displaystyle \frac{1}{Mx - Ny}\) is an IF of the equation.
\end{enumerate}
\bibliographystyle{unsrt}
\bibliography{main}
\end{document}