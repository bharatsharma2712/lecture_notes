\documentclass[oneside]{book}
\usepackage[a4paper, margin=3cm]{geometry}
\usepackage{import}
\usepackage{xcolor}
\usepackage{pdfpages}
\usepackage{transparent}
\usepackage{subcaption}
\usepackage{float}
\usepackage{amsmath}
\usepackage{amssymb}
\usepackage{hyperref}
\usepackage{babel}
\usepackage{cleveref}

\newcommand{\incfig}[1]{%
    \import{./figures/}{#1.pdf_tex}
}
\newcommand{\incfigsc}[2][1]{%
    \def\svgwidth{#1\columnwidth}
    \import{./figures/}{#2.pdf_tex}
}

\newcommand{\halfsubfig}[2][1]{
	\begin{subfigure}{0.45\linewidth}
		\centering
		\incfig{#1}
		\caption{#2}
	\end{subfigure}
}

% \newcommand{\j}{\mathrm{j}}
\hypersetup{
    colorlinks,
    linkcolor={blue!80!black},
    citecolor={blue!50!black},
    urlcolor={blue!80!black}
}

\title{EE201 - Signals and Systems}
\author{Indian Institute of Technology Ropar}
\begin{document}
\maketitle
{
	\hypersetup{
		hidelinks
	}
	\tableofcontents
}
\renewcommand{\arraystretch}{1.5}%Sets the vertical margin

\chapter{Analysis of Signals and Systems}
\section{Energy and Power}
\begin{table}[ht]
	\centering
	\begin{tabular}{|r|cc|}
		\hline
		                & \textbf{Continuous Time Signal}, $x(t)$                   & \textbf{Discrete Time Signal}, $x[n]$                       \\
		\hline
		\textbf{Energy} & \(\displaystyle E_x = \int\limits_{t_1}^{t_2}|x(t)|^2dt\) & \(\displaystyle E_x = \sum\limits_{n = N_1}^{N_2}|x[n]|^2\) \\
		\textbf{Power}  & \(\displaystyle E_x = \frac{E_x}{t_2 - t_1}\)             & \(\displaystyle P_x = \frac{E_x}{N_2 - N_1 + 1}\)           \\
		\hline
	\end{tabular}
\end{table}
In infinite duration signals (\(-\infty<t<+\infty, -\infty<n<+\infty\))
\begin{table}[ht]
	\centering
	\begin{tabular}{|r|cc|}
		\hline
		                & \textbf{Continuous Time Signal}, $x(t)$                                              & \textbf{Discrete Time Signal}, $x[n]$                                                  \\
		\hline
		\textbf{Energy} & \(\displaystyle E_\infty = \lim_{T\rightarrow\infty}\int\limits_{-T}^{T}|x(t)|^2dt\) & \(\displaystyle E_\infty = \lim_{N\rightarrow\infty}\sum\limits_{n = -N}^{N}|x[n]|^2\) \\
		\textbf{Power}  & \(\displaystyle E_\infty = \lim_{T\rightarrow\infty}\frac{E_\infty}{2T}\)            & \(\displaystyle P_\infty = \lim_{N\rightarrow\infty}\frac{E_\infty}{2N + 1}\)          \\
		\hline
	\end{tabular}
\end{table}

\section{Periodic Signals}
A periodic signal repeats itself after sometime.
\subsubsection{Continuous signals}
If
\(
x(t) = x(t + T)\ \forall t
\), \(T>0\)
then the signal is periodic. The minimum value of $T$ that satisfies the condition is calle the \textit{fundamental period}.
\subsubsection{Discrete signals}
If
\(
x[n] = x[n + N]\ \forall n
\), \(N>0, N\in\mathbb{Z}\)
then the signal is periodic. The minimum value of $N$ that satisfies the condition is calle the \textit{fundamental period}.

\section{Exponential Signals}
\subsection{Continuous Time Exponential Signals}
A continuous time exponential signal can be represented as
\begin{equation}
	x(t) = Ce^{at}
\end{equation}
where \(C, a \in \mathbb{C}\).
\\\\
\noindent \cite{continuous_time_exponential_general_case}Consider the general case where \(C = |C|e^{j\theta}\) and \(a = \sigma + \mathsf{j}\omega\). The signal becomes
\[
	x(t) = |C|e^{\sigma t}e^{\mathsf{j}(wt + \theta)}
\]
\begin{itemize}
	\item $\omega$ is angular frequency. If \(\omega \neq 0\), the signal is sinusoidal, with \(f = \frac{\omega}{2\pi}\).
	\item $\sigma$ is the attenuation factor. If \(\sigma \neq 0\), the signal decays/grows and is bounded by the envelope \(|C|e^{\sigma t}\).
	\item $\theta$ is the initial phase.
\end{itemize}

\subsection{Discrete Time Exponential Signals}
A discrete time exponential signal can be represented as
\begin{equation}
	x[n] = Ce^{an}
\end{equation}
where \(C, a \in \mathbb{C}\).
\\\\
\noindent Consider the general case where \(C = |C|e^{j\theta}\) and \(a = \sigma + \mathsf{j}\omega\). The signal becomes
\[
	x[n] = |C|e^{\sigma n}e^{\mathsf{j}(wn + \theta)}
\]
\begin{itemize}
	\item $\omega$ is \textbf{related to} angular frequency. If \(\omega \neq 2n\pi, n\in\mathbb{Z}\), the signal is periodic.\\
	      Unlike continuous-time signals, the angular frequency cannot take all values. The range is \([0, 2\pi)\) or \([-\pi, \pi)\) because
	      \[
		      e^{\mathsf{j}(\omega + 2\pi)n} = e^{\mathsf{j}\omega n}e^{\mathsf{j}2n\pi} = e^{\mathsf{j}\omega n}
	      \]
	      Rate of oscillation is small for $\omega\sim 2n\pi$, and high for $\omega\sim (2n + 1)\pi, n\in\mathbb{Z}$.
	\item $\sigma$ is the attenuation factor. If \(\sigma \neq 0\), the signal decays/grows and is bounded by the envelope \(|C|e^{\sigma n}\).
	\item $\theta$ is the initial phase.
\end{itemize}

\subsection{Periodicity}
\subsubsection{Continuous-time exponential signals}
For a continuous-time exponential signal to be periodic
\begin{align*}
	e^{\mathsf{j}\omega(t + T)}        & = e^{\mathsf{j}\omega t} \\
	\Rightarrow e^{\mathsf{j}\omega T} & = 1                      \\
	\Rightarrow \omega T               & = 2m\pi, m\in\mathbb{Z}  \\
\end{align*}
Therefore, continuous-time exponential signal is periodic for all $\omega>0$ and their \textit{fundamental time period} is
\[
	T = \frac{2\pi}{\omega}
\]

\subsubsection{Discrete-time exponential signals}
For a discrete-time exponential signal to be periodic
\begin{align*}
	e^{\mathsf{j}\omega(n + N)}        & = e^{\mathsf{j}\omega n} \\
	\Rightarrow e^{\mathsf{j}\omega N} & = 1                      \\
	\Rightarrow \omega N               & = 2m\pi, m\in\mathbb{Z}  \\
\end{align*}
Therefore, discrete-time exponential signal is periodic only if \(\displaystyle \frac{\omega}{2\pi} = \frac{m}{N}\) i.e. \(\displaystyle\frac{w}{2\pi}\) is a rational number.\\
The \textit{fundamental period} is
\[
	N = m\frac{2\pi}{\omega}
\]
where $m$ is smallest positive integer such that $N$ evaluates to an integer.

\subsection{Harmonics}
\subsubsection{Continuous-time exponential signals}
\[
	\phi_k(t) = e^{\mathsf{j}k[\frac{2\pi}{T}]t} = e^{\mathsf{j}k\omega t}
\]
\subsubsection{Discrete-time exponential signals}
\[
	\phi_k[n] = e^{\mathsf{j}k[\frac{2\pi}{N}]n} = e^{\mathsf{j}k\frac{\omega}{m} n}
\]

\section{Unit Impulse and Step Functions}
\subsection{Discrete-Time Case}
\subsubsection{Unit Impulse}
\[
	\delta[n] = \begin{cases}
		0 & n\neq 0 \\
		1 & n = 0
	\end{cases}
\]
\begin{figure}[ht]
	\centering
	\incfig{delta_n}
	\caption{Unit impulse function in discrete-time}
\end{figure}
\subsubsection{Unit Step}
\[
	U[n] = \begin{cases}
		0 & n < 0    \\
		1 & n \geq 0
	\end{cases}
\]
\begin{figure}[ht]
	\centering
	\incfig{u_n}
	\caption{Unit step function in discrete-time}
\end{figure}
\subsubsection{Properties}
\begin{enumerate}
	\item \(\delta[n] = U[n] - U[n-1]\)
	\item \(U[n] = \sum\limits_{m = -\infty}^{n}\delta[n] = \sum\limits_{k = 0}^{\infty}\delta[n-k]\)
	\item \textbf{Sampling Property of unit impulse function}
	      \begin{align*}
		      x[n]\cdot\delta[n]     & = x[0]\cdot\delta[n]         \\
		      x[n]\cdot\delta[n-n_0] & = x[n_0]\cdot\delta[n - n_0]
	      \end{align*}
\end{enumerate}
\subsection{Continuous-Time Case}
\subsubsection{Unit Step}
\[
	U(t) = \begin{cases}
		0 & t < 0 \\
		1 & t > 0
	\end{cases}
\]
\begin{figure}[ht]
	\centering
	\incfig{u_t}
	\caption{Unit step function in continuous-time}
\end{figure}
This function is discontinuous at $t = 0$. It is modified a little bit to make it continuous.
\[
	U_\Delta(t) = \begin{cases}
		0                & t<0                 \\
		\frac{t}{\Delta} & 0\leq t \leq \Delta \\
		1                & t > \Delta
	\end{cases}
\]
\begin{figure}[ht]
	\centering
	\incfig{u_delta_t}
	\caption{Continuous unit step function in continuous-time}
\end{figure}

\subsubsection{Unit Impulse}
\begin{align*}
	U(t)      & = \int\limits_{-\infty}^{t}\delta(\tau)d\tau \\
	\delta(t) & = \frac{dU(t)}{dt}                           \\
\end{align*}
Since, \(U(t)\) is not differentiable.
\begin{align*}
	\delta(t) & = \lim_{\Delta\rightarrow 0}\frac{dU_\Delta(t)}{dt} \\
	\delta(t) & = \begin{cases}
		0                & t<0                  \\
		\frac{1}{\Delta} & 0 \leq t \leq \Delta \\
		0                & t>\Delta
	\end{cases}
\end{align*}
\begin{figure}[ht]
	\centering
	\incfig{delta_t}
	\caption{Unit impulse function in continuous-time}
\end{figure}
\subsubsection{Properties}
\begin{enumerate}
	\item \(\displaystyle\int\limits_{-\infty}^{+\infty}\delta(\tau)d\tau = 1\)
	\item \textbf{Sampling Property of unit impulse function}
	      \begin{align*}
		      x(t)\cdot\delta(t)     & = x(0)\cdot\delta(t)         \\
		      x(t)\cdot\delta(t-t_0) & = x(t_0)\cdot\delta(t - t_0)
	      \end{align*}
\end{enumerate}

\section{Systems}
System is the interconnection of components/devices/subsystems. A \textit{continuous-time system} is represented by \textit{differential equation} and a \textit{discrete-time system} is represented by \textit{difference equation}.

\subsection{Interconnection of Systems}
\begin{enumerate}
	\item \textbf{Series (Cascade) Connection}
	      \begin{figure}[ht]
		      \centering
		      \incfig{series_interconnection}
		      \caption{Example of series interconnection}
	      \end{figure}
	\item \textbf{Parallel Connection}
	      \begin{figure}[ht]
		      \centering
		      \incfig{parallel_interconnection}
		      \caption{Example of parallel interconnection}
	      \end{figure}
	\item \textbf{Series-Parallel Connection}\\
	      It is combination of series and parallel connections.
	\item \textbf{Feedback Interconnection}\\
	      In this kind of interconnection, some part of output is again fed as input to the system.
	      \begin{figure}[ht]
		      \centering
		      \incfig{feedback_interconnection}
		      \caption{Example of feedback interconnection}
	      \end{figure}
\end{enumerate}

\subsection{Basic Properties}
Some basic properties of a system are
\begin{enumerate}
	\item Memory
	\item Invertibility
	\item Causality
	\item Stability
	\item Time-invariance
	\item Linearity
\end{enumerate}

\bibliographystyle{unsrt}
\bibliography{main}
\end{document}