\documentclass[oneside]{book}
\usepackage[a4paper, margin=3cm]{geometry}
\usepackage{import}
\usepackage{xcolor}
\usepackage{pdfpages}
\usepackage{transparent}
\usepackage{subcaption}
\usepackage{float}
\usepackage{amsmath}
\usepackage{amssymb}
\usepackage{hyperref}
\usepackage{babel}
\usepackage{cleveref}

\newcommand{\incfig}[1]{%
    \import{./figures/}{#1.pdf_tex}
}
\newcommand{\incfigsc}[2][1]{%
    \def\svgwidth{#1\columnwidth}
    \import{./figures/}{#2.pdf_tex}
}

\newcommand{\halfsubfig}[2][1]{
	\begin{subfigure}{0.45\linewidth}
		\centering
		\incfig{#1}
		\caption{#2}
	\end{subfigure}
}

% \newcommand{\j}{\mathrm{j}}
\hypersetup{
    colorlinks,
    linkcolor={blue!80!black},
    citecolor={blue!50!black},
    urlcolor={blue!80!black}
}

\title{EE201 - Signals and Systems}
\author{Indian Institute of Technology Ropar}
\begin{document}
\maketitle
{
	\hypersetup{
		hidelinks
	}
	\tableofcontents
}
\renewcommand{\arraystretch}{1.5}%Sets the vertical margin

\chapter{Analysis of Signals and Systems}
\section{Energy and Power}
\begin{table}[ht]
	\centering
	\begin{tabular}{|r|cc|}
		\hline
		                & \textbf{Continuous Time Signal}, $x(t)$                   & \textbf{Discrete Time Signal}, $x[n]$                       \\
		\hline
		\textbf{Energy} & \(\displaystyle E_x = \int\limits_{t_1}^{t_2}|x(t)|^2dt\) & \(\displaystyle E_x = \sum\limits_{n = N_1}^{N_2}|x[n]|^2\) \\
		\textbf{Power}  & \(\displaystyle E_x = \frac{E_x}{t_2 - t_1}\)             & \(\displaystyle P_x = \frac{E_x}{N_2 - N_1 + 1}\)           \\
		\hline
	\end{tabular}
\end{table}
In infinite duration signals (\(-\infty<t<+\infty, -\infty<n<+\infty\))
\begin{table}[ht]
	\centering
	\begin{tabular}{|r|cc|}
		\hline
		                & \textbf{Continuous Time Signal}, $x(t)$                                              & \textbf{Discrete Time Signal}, $x[n]$                                                  \\
		\hline
		\textbf{Energy} & \(\displaystyle E_\infty = \lim_{T\rightarrow\infty}\int\limits_{-T}^{T}|x(t)|^2dt\) & \(\displaystyle E_\infty = \lim_{N\rightarrow\infty}\sum\limits_{n = -N}^{N}|x[n]|^2\) \\
		\textbf{Power}  & \(\displaystyle E_\infty = \lim_{T\rightarrow\infty}\frac{E_\infty}{2T}\)            & \(\displaystyle P_\infty = \lim_{N\rightarrow\infty}\frac{E_\infty}{2N + 1}\)          \\
		\hline
	\end{tabular}
\end{table}

\section{Periodic Signals}
A periodic signal repeats itself after sometime.
\subsubsection{Continuous signals}
If
\(
x(t) = x(t + T)\ \forall t
\), \(T>0\)
then the signal is periodic. The minimum value of $T$ that satisfies the condition is calle the \textit{fundamental period}.
\subsubsection{Discrete signals}
If
\(
x[n] = x[n + N]\ \forall n
\), \(N>0, N\in\mathbb{Z}\)
then the signal is periodic. The minimum value of $N$ that satisfies the condition is calle the \textit{fundamental period}.

\section{Exponential Signals}
\subsection{Continuous Time Exponential Signals}
A continuous time exponential signal can be represented as
\begin{equation}
	x(t) = Ce^{at}
\end{equation}
where \(C, a \in \mathbb{C}\).
\\\\
\noindent \cite{continuous_time_exponential_general_case}Consider the general case where \(C = |C|e^{j\theta}\) and \(a = \sigma + \mathsf{j}\omega\). The signal becomes
\[
	x(t) = |C|e^{\sigma t}e^{\mathsf{j}(wt + \theta)}
\]
\begin{itemize}
	\item $\omega$ is angular frequency. If \(\omega \neq 0\), the signal is sinusoidal, with \(f = \frac{\omega}{2\pi}\).
	\item $\sigma$ is the attenuation factor. If \(\sigma \neq 0\), the signal decays/grows and is bounded by the envelope \(|C|e^{\sigma t}\).
	\item $\theta$ is the initial phase.
\end{itemize}

\subsection{Discrete Time Exponential Signals}
A discrete time exponential signal can be represented as
\begin{equation}
	x[n] = Ce^{an}
\end{equation}
where \(C, a \in \mathbb{C}\).
\\\\
\noindent Consider the general case where \(C = |C|e^{j\theta}\) and \(a = \sigma + \mathsf{j}\omega\). The signal becomes
\[
	x[n] = |C|e^{\sigma n}e^{\mathsf{j}(wn + \theta)}
\]
\begin{itemize}
	\item $\omega$ is \textbf{related to} angular frequency. If \(\omega \neq 2n\pi, n\in\mathbb{Z}\), the signal is periodic.\\
	      Unlike continuous-time signals, the angular frequency cannot take all values. The range is \([0, 2\pi)\) or \([-\pi, \pi)\) because
	      \[
		      e^{\mathsf{j}(\omega + 2\pi)n} = e^{\mathsf{j}\omega n}e^{\mathsf{j}2n\pi} = e^{\mathsf{j}\omega n}
	      \]
	      Rate of oscillation is small for $\omega\sim 2n\pi$, and high for $\omega\sim (2n + 1)\pi, n\in\mathbb{Z}$.
	\item $\sigma$ is the attenuation factor. If \(\sigma \neq 0\), the signal decays/grows and is bounded by the envelope \(|C|e^{\sigma n}\).
	\item $\theta$ is the initial phase.
\end{itemize}

\subsection{Periodicity}
\subsubsection{Continuous-time exponential signals}
For a continuous-time exponential signal to be periodic
\begin{align*}
	e^{\mathsf{j}\omega(t + T)}        & = e^{\mathsf{j}\omega t} \\
	\Rightarrow e^{\mathsf{j}\omega T} & = 1                      \\
	\Rightarrow \omega T               & = 2m\pi, m\in\mathbb{Z}  \\
\end{align*}
Therefore, continuous-time exponential signal is periodic for all $\omega>0$ and their \textit{fundamental time period} is
\[
	T = \frac{2\pi}{\omega}
\]

\subsubsection{Discrete-time exponential signals}
For a discrete-time exponential signal to be periodic
\begin{align*}
	e^{\mathsf{j}\omega(n + N)}        & = e^{\mathsf{j}\omega n} \\
	\Rightarrow e^{\mathsf{j}\omega N} & = 1                      \\
	\Rightarrow \omega N               & = 2m\pi, m\in\mathbb{Z}  \\
\end{align*}
Therefore, discrete-time exponential signal is periodic only if \(\displaystyle \frac{\omega}{2\pi} = \frac{m}{N}\) i.e. \(\displaystyle\frac{w}{2\pi}\) is a rational number.\\
The \textit{fundamental period} is
\[
	N = m\frac{2\pi}{\omega}
\]
where $m$ is smallest positive integer such that $N$ evaluates to an integer.

\subsection{Harmonics}
\subsubsection{Continuous-time exponential signals}
\[
	\phi_k(t) = e^{\mathsf{j}k[\frac{2\pi}{T}]t} = e^{\mathsf{j}k\omega t}
\]
\subsubsection{Discrete-time exponential signals}
\[
	\phi_k[n] = e^{\mathsf{j}k[\frac{2\pi}{N}]n} = e^{\mathsf{j}k\frac{\omega}{m} n}
\]

\section{Unit Impulse and Step Functions}
\subsection{Discrete-Time Case}
\subsubsection{Unit Impulse}
\[
	\delta[n] = \begin{cases}
		0 & n\neq 0 \\
		1 & n = 0
	\end{cases}
\]
\begin{figure}[ht]
	\centering
	\incfig{delta_n}
	\caption{Unit impulse function in discrete-time}
\end{figure}
\subsubsection{Unit Step}
\[
	u[n] = \begin{cases}
		0 & n < 0    \\
		1 & n \geq 0
	\end{cases}
\]
\begin{figure}[ht]
	\centering
	\incfig{u_n}
	\caption{Unit step function in discrete-time}
\end{figure}
\subsubsection{Properties}
\begin{enumerate}
	\item \(\delta[n] = u[n] - u[n-1]\)
	\item \(u[n] = \sum\limits_{m = -\infty}^{n}\delta[n] = \sum\limits_{k = 0}^{\infty}\delta[n-k]\)
	\item \textbf{Sampling Property of unit impulse function}
	      \begin{align*}
		      x[n]\cdot\delta[n]     & = x[0]\cdot\delta[n]         \\
		      x[n]\cdot\delta[n-n_0] & = x[n_0]\cdot\delta[n - n_0]
	      \end{align*}
\end{enumerate}
\subsection{Continuous-Time Case}
\subsubsection{Unit Step}
\[
	u(t) = \begin{cases}
		0 & t < 0 \\
		1 & t > 0
	\end{cases}
\]
\begin{figure}[ht]
	\centering
	\incfig{u_t}
	\caption{Unit step function in continuous-time}
\end{figure}
This function is discontinuous at $t = 0$. It is modified a little bit to make it continuous.
\[
	U_\Delta(t) = \begin{cases}
		0                & t\leq0        \\
		\frac{t}{\Delta} & 0< t < \Delta \\
		1                & t \geq \Delta
	\end{cases}
\]
\begin{figure}[ht]
	\centering
	\incfig{u_delta_t}
	\caption{Continuous unit step function in continuous-time}
\end{figure}

\subsubsection{Unit Impulse}
\begin{align*}
	u(t)      & = \int\limits_{-\infty}^{t}\delta(\tau)d\tau \\
	\delta(t) & = \frac{du(t)}{dt}                           \\
\end{align*}
Since, \(u(t)\) is not differentiable.
\begin{align*}
	\delta(t) & = \lim_{\Delta\rightarrow 0}\frac{dU_\Delta(t)}{dt} \\
	\delta(t) & = \begin{cases}
		0                & t\leq0         \\
		\frac{1}{\Delta} & 0 < t < \Delta \\
		0                & t\geq\Delta
	\end{cases}
\end{align*}
\begin{figure}[ht]
	\centering
	\incfig{delta_t}
	\caption{Unit impulse function in continuous-time}
\end{figure}
\subsubsection{Properties}
\begin{enumerate}
	\item \(\displaystyle\int\limits_{-\infty}^{+\infty}\delta(\tau)d\tau = 1\)
	\item \textbf{Sampling Property of unit impulse function}
	      \begin{align*}
		      x(t)\cdot\delta(t)     & = x(0)\cdot\delta(t)         \\
		      x(t)\cdot\delta(t-t_0) & = x(t_0)\cdot\delta(t - t_0)
	      \end{align*}
\end{enumerate}

\section{Systems}
System is the interconnection of components/devices/subsystems. A \textit{continuous-time system} is represented by \textit{differential equation} and a \textit{discrete-time system} is represented by \textit{difference equation}.

\subsection{Interconnection of Systems}
\begin{enumerate}
	\item \textbf{Series (Cascade) Connection}
	      \begin{figure}[ht]
		      \centering
		      \incfig{series_interconnection}
		      \caption{Example of series interconnection}
	      \end{figure}
	\item \textbf{Parallel Connection}
	      \begin{figure}[ht]
		      \centering
		      \incfig{parallel_interconnection}
		      \caption{Example of parallel interconnection}
	      \end{figure}
	\item \textbf{Series-Parallel Connection}\\
	      It is combination of series and parallel connections.
	\item \textbf{Feedback Interconnection}\\
	      In this kind of interconnection, some part of output is again fed as input to the system.
	      \begin{figure}[ht]
		      \centering
		      \incfig{feedback_interconnection}
		      \caption{Example of feedback interconnection}
	      \end{figure}
\end{enumerate}

\section{Basic Properties of a System}
Some basic properties of a system are
\begin{enumerate}
	\item Memory
	\item Invertibility
	\item Causality
	\item Stability
	\item Time-invariance
	\item Linearity
\end{enumerate}

In all examples below, $x(t)$(or $x[n]$) is the input to the system and the output is $y(t)$(or $y[n]$).

\subsection{Memory}
Systems can be divided into two classes on basis of memory:
\begin{enumerate}
	\item \textbf{Memoryless:} System whose output depends only on present input.\\
	      \textsc{Examples}
	      \begin{itemize}
		      \item \(y[n] = (2x[n] - x^2[n])^2\)
		      \item \(y(t) = Rx(t)\)
	      \end{itemize}
	\item \textbf{With memory:} System whose output depends on past/future value also.\\
	      \textsc{Examples}
	      \begin{itemize}
		      \item \(y[n] = x[n-1]\)
		      \item \(y[n] = \sum\limits_{k = -\infty}^n x[k]\)
		      \item \(y(t) = \frac{1}{c}\int\limits_{-\infty}^{\infty}x(\tau)d\tau\)
	      \end{itemize}
\end{enumerate}

\subsection{Invertibility}
Systems can be divided into two classes on basis of invertibility:
\begin{enumerate}
	\item \textbf{Invertible:} Systems for which distinct input leads to distinct outputs (i.e. they are one-one and onto).\\
	      \textsc{Examples}
	      \begin{itemize}
		      \item \(y(t) = 2 x(t)\)
	      \end{itemize}
	      When a system and its inverse system are cascaded then output is same as input and such group of system is called \textit{identity systems} (\(x(t) = t\)).
	\item \textbf{Non-invertible:} Systems which are not invertible.\\
	      \textsc{Examples}
	      \begin{itemize}
		      \item \(y(t) = c\)
	      \end{itemize}
\end{enumerate}

\subsection{Causality}
Systems can be divided into two classes on basis of causality:
\begin{enumerate}
	\item \textbf{Causal/Non-anticipative:} Systems whose output is dependent only on present and past (but not future) values of the input.\\
	      \textsc{Examples}
	      \begin{itemize}
		      \item All memoryless systems(they use only present input)
	      \end{itemize}
	      All practical systems are causal, unless they use recorded signals as future values.\\\\
	      \noindent\textbf{Note: } \textit{Causal signals} are signals which start after $t=0$, \textit{non-causal signals} are signals that start before $t=0$ and \textit{anti-causal signals} are signals that end after $t=0$.
	\item \textbf{Non-causal system:} Systems which are not causal.\\
	      \textsc{Examples}
	      \begin{itemize}
		      \item \(y(t) = x(t+1)\)
	      \end{itemize}
\end{enumerate}

\subsection{Stablility}
Systems which produce bounded output for bounded input are called \textit{stable system}. Such systems are are called BIBO(bounded input, bounded output) stable.\\
\textsc{Examples}
\begin{itemize}
	\item Charging/discharging capacitor
\end{itemize}
\noindent\textbf{Note: } A signal is called bounded if \(\exists\ B > 0\) such that the signal magnitude never exceeds \(B\). \cite{bibo_stability_wiki}

\subsection{Time Invariance}
Systems can be divided into two classes on basis of time-invariance:
\begin{enumerate}
	\item \textbf{Time invariant:} System whose output does not depend on the instant or time of applying the input. Delay in input produces same delay in output.\\
	      \textsc{Examples}
	      \begin{itemize}
		      \item \(y(t) = x(t)\)
	      \end{itemize}
	\item \textbf{Time variant:} Systems whose output depend on the time of application of input.\\
	      \textsc{Examples}
	      \begin{itemize}
		      \item \(y(t) = tx(t)\),
	      \end{itemize}
\end{enumerate}

\subsection{Linearity}
Systems can be divided into two classes on basis of linearity:
\begin{enumerate}
	\item \textbf{Linear:} Systems that can be superimposed i.e. they are additive and homogeneous.\\
	      \textsc{Examples}
	      \begin{itemize}
		      \item \(y(t) = x(t_0)\)
	      \end{itemize}
	      \textbf{Additivity:} If \(x_1(t) \rightarrow y_1(t)\) and \(x_2(t) \rightarrow y_2(t)\), then
	      \[
		      x_1(t) + x_2(t) \rightarrow y_1(t) + y_2(t)
	      \]
	      \textbf{Homogenity:} If \(x(t) \rightarrow y(t)\), then
	      \[
		      ax(t) \rightarrow ay(t)\quad (a\in\mathbb{C})
	      \]
	\item \textbf{Non-linear:} Systems which do not follow additivity or homogenity or both.\\
	      \textsc{Examples}
	      \begin{itemize}
		      \item \(y(t) = x^2(t) + c\)
	      \end{itemize}
\end{enumerate}

\section{Linear Time Invariant(LTI) Systems}
Unit impulse signal will be used as a basis for creating other signals. In other words, we will be writing signals as a linear combination of shifted unit-impulse signals.
\subsection{Discrete-Time LTI Systems}
Consider an arbitrary signal $x[n]$, it can be represented as linear combination of shifted unit-impulse signals.
\begin{align}
	x[n] = \sum\limits_{k = -\infty}^{\infty}x[k]\delta[n-k]
	\label{discrete_signal_as_sum_of_unit_impulses}
\end{align}
This is called \textit{sifting property} of impulse in discrete-time case.
\\\\
\noindent Let $h_k[n]$ be the impulse response (system response to the unit-impulse $\delta[n - k]$). ($h_0 = h$)\\\\
\noindent\(\because\) System is linear\\
\(\therefore\) From \cref{discrete_signal_as_sum_of_unit_impulses}, combining the system response to different unit impulses will give the output
\begin{align}
	y[n] = \sum\limits_{k = -\infty}^{\infty}x[k]h_k[n]
	\label{discrete_response_of_system_using_linear_property}
\end{align}
\noindent\(\because\) System is time-invariant\\
\(\therefore h_k[n] = h[n-k]\)\\
The \cref{discrete_response_of_system_using_linear_property}, reduces to
\begin{align}
	y[n] = \sum\limits_{k = -\infty}^{\infty}x[k]h[n-k]
	\label{discrete_response_of_system_using_time_invariant_property}
\end{align}
Using convolution sum\cite{convolution_wiki}, \cref{discrete_response_of_system_using_time_invariant_property} reduces to
\begin{align*}
	y[n] = x[n]*h[n]
\end{align*}

\subsection{Continuous-Time LTI Systems}
Like discrete-time case, we can write any arbitrary signal $x(t)$ as a linear combination of shifted unit-impulse signals.
\begin{align}
	x(t) = \int\limits_{-\infty}^{\infty}x(\tau)\delta(t - \tau)d\tau
	\label{continuous_signal_as_sum_of_unit_impulses}
\end{align}
This is called \textit{sifting property} of impulse in continuous-time case.
\\\\
\noindent Let $h_\tau(t)$ be the impulse response (system response to the unit-impulse $\delta(t - \tau)$). ($h_0 = h$)\\\\
\noindent\(\because\) System is linear\\
\(\therefore\) From \cref{continuous_signal_as_sum_of_unit_impulses}, combining the system response to different unit impulses will give the output
\begin{align}
	y(t) = \int\limits_{-\infty}^{\infty}x(\tau)h_\tau(t)d\tau
	\label{continuous_response_of_system_using_linear_property}
\end{align}
\noindent\(\because\) System is time-invariant\\
\(\therefore h_\tau(t) = h(t-\tau)\)\\
The \cref{continuous_response_of_system_using_linear_property}, reduces to
\begin{align}
	y(t) = \int\limits_{-\infty}^{\infty}x(\tau)h(t - \tau)d\tau
	\label{continuous_response_of_system_using_time_invariant_property}
\end{align}
Using convolution integral\cite{convolution_wiki}, \cref{continuous_response_of_system_using_time_invariant_property} reduces to
\begin{align*}
	y(t) = x(t)*h(t)
\end{align*}

\textbf{Corollary:} Impulse response is complete characterization of LTI system.

\subsection{Properties of LTI Systems}
The properties given below are for LTI systems only.
\begin{enumerate}
	\item \textbf{Commutative Property}
	      \[
		      y(t) = x(t) * h(t) = h(t) * x(t)
	      \]
	      \[
		      y[n] = x[n] * h[n] = h[n] * x[n]
	      \]
	\item \textbf{Distributive Property}
	      \begin{align*}
		      x(t)*[h_1(t) + h_2(t)] & = x(t)*h_1(t) + x(t)*h_2(t) \\
		      x[n]*[h_1[n] + h_2[n]] & = x[n]*h_1[n] + x[n]*h_2[n]
	      \end{align*}
	\item \textbf{Associative Property}
	      \begin{align*}
		      x[n]*[h_1[n] *h_2[n]] & = [x[n]*h_1[n]]*h_2[n] \\
		      x(t)*[h_1(t) *h_2(t)] & = [x(t)*h_1(t)]*h_2(t)
	      \end{align*}
	\item When LTI system is memoryless\\
	      \(h[n] = 0 \text{ for } n \neq 0\) and \(h(t) = 0 \text{ for } t \neq 0\)
	      \begin{align*}
		      h(t) & = k\delta(t) & y(t) & = kx(t) \\
		      h[n] & = k\delta[n] & y[n] & = kx[n]
	      \end{align*}
	      \textbf{Note: } \(x(t)*\delta(t) = x(t)\) and \(x[n]*\delta[n] = x[n]\)
	\item LTI system \(h_1(t)\)\textcolor{gray}{(or \(h_1[n]\))} is invertible if \(\exists\ h_2(t)\)\textcolor{gray}{(or \(h_2[n]\))}, such that
	      \begin{align*}
		      h_1(t)*h_2(t)                                & = \delta(t)                   \\
		      \textcolor{gray}{\text{or } h_1[n] * h_2[n]} & \textcolor{gray}{= \delta[n]}
	      \end{align*}
	      and \(h_2(t)\)\textcolor{gray}{(or \(h_2[n]\))} is its inverse.
	\item When LTI system is causal
	      \begin{align*}
		      y[n] & = \sum\limits_{k = -\infty}^{\infty} x[k]h[n-k]                          \\
		           & = \sum\limits_{k = -\infty}^n x[k]h[n-k]\quad{(\because\ x[k] = 0, k>n)} \\
		           & = \sum\limits_{k = 0}^\infty x[n-k]h[k]
	      \end{align*}
	      \begin{align*}
		      y(t) & = \int\limits_{-\infty}^{\infty} x(\tau)h(t-\tau)d\tau                                \\
		           & = \int\limits_{-\infty}^t x(\tau)h(t-\tau)d\tau\quad{(\because\ x(\tau) = 0, \tau>t)} \\
		           & = \int\limits_{0}^\infty x(t-\tau)h(\tau)
	      \end{align*}
	      So, we can also say that for a causal \textbf{system}
	      \begin{align*}
		      h(t) & = 0\ \forall\ t<0 \\
		      h[n] & = 0\ \forall\ n<0
	      \end{align*}
	      Similarly, for a causal \textbf{signal}
	      \begin{align*}
		      x(t) & = 0\ \forall\ t<0 \\
		      x[n] & = 0\ \forall\ n<0
	      \end{align*}
	      \textbf{Initial test: } A causal system gives no output unless an input is given.
	\item When LTI system is stable
	      \begin{align*}
		      |x[n]| & \leq B\quad \forall\ n                                     \\
		      |y[n]| & = \bigl|\sum\limits_{k = -\infty}^{\infty}x[n]h[n-k]\bigl| \\
		      |y[n]| & \leq \sum\limits_{k = -\infty}^{\infty}|x[n]|\cdot|h[n-k]| \\
		      |y[n]| & \leq B\sum\limits_{k = -\infty}^{\infty}|\cdot|h[n-k]|     \\
	      \end{align*}
	      Therefore, for a system to be stable the `impulse response must be absolutely integrable'.
	      \begin{align*}
		      \sum\limits_{k = -\infty}^{\infty}|h[k]|     & < \infty \\
		      \int\limits_{-\infty}^{\infty}|h(\tau)|d\tau & < \infty \\
	      \end{align*}
\end{enumerate}

\subsection{Unit Step Response}
Unit step response(response to a unit step signal) is denoted by \(s[n]\) or \(s(t)\). It can be calculated using impulse response.
\begin{align*}
	s[n] & = u[n]*h[n]                             \\
	     & = \sum\limits_{k = -\infty}^{n}h[k]     \\
	s(t) & = u(t)*h(t)                             \\
	     & = \int\limits_{-\infty}^{t}h(\tau)d\tau
\end{align*}
Similarly, impulse response can be calculated from step response.
\begin{align*}
	h[n] & = s[n] - s[n-1]                  \\
	h(t) & = \frac{ds(t)}{dt} = s^\prime(t)
\end{align*}

\subsection{Differential Equation}
An LTI system, in continuous-time case, can be represented using \textbf{linear} differential equation with \textbf{constant coefficients}. A general \(N^{th}\) order linear constant coefficient differential equation is
\[
	\sum\limits_{k = 0}^{N}a_k\frac{d^ky(t)}{dt^k} = \sum\limits_{k = 0}^{M}b_k\frac{d^kx(t)}{dt^k}
\]
When \(N = 0\), we get an explicit relation. In other cases, the relation between input and output is implicit.
\subsubsection{General solution}
\[
	y(t) = \underbrace{y_p(t)}\limits_{\text{Particular solution}} + \underbrace{y_h(t)}\limits_{\text{Homogeneous solution}}
\]

\subsection{Difference Equation}
An LTI system, in discrete-time case, can be represented using \textbf{linear} difference equation with \textbf{constant coefficients}. A general \(N^{th}\) order linear constant coefficient difference equation is
\[
	\sum\limits_{k = 0}^{N}a_ky[n-k] = \sum\limits_{k = 0}^{M}b_kx[n-k]
\]
When \(N = 0\), we get an explicit relation and the system has finite impulse responses(FIR). In other cases, the relation between input and output is implicit/recursive and the system has infinite impulse responses (IIR).


\bibliographystyle{unsrt}
\bibliography{main}
\end{document}