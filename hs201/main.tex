\documentclass[oneside]{book}
\usepackage[a4paper, margin=3cm]{geometry}
\usepackage{import}
\usepackage{xcolor}
\usepackage{multirow}
\usepackage{pdfpages}
\usepackage{transparent}
\usepackage{subcaption}
\usepackage{float}
\usepackage{amsmath}
\usepackage{amssymb}
\usepackage{hyperref}
\usepackage{babel}
\usepackage{cleveref}

\newcommand{\incfig}[1]{%
    \import{./figures/}{#1.pdf_tex}
}
\newcommand{\incfigsc}[2][1]{%
    \def\svgwidth{#1\columnwidth}
    \import{./figures/}{#2.pdf_tex}
}

\newcommand{\halfsubfig}[2][1]{
	\begin{subfigure}{0.45\linewidth}
		\centering
		\incfig{#1}
		\caption{#2}
	\end{subfigure}
}
\hypersetup{
    colorlinks,
    linkcolor={blue!80!black},
    citecolor={blue!50!black},
    urlcolor={blue!80!black}
}

\title{HS201 - Economics}
\author{Indian Institute of Technology Ropar}
\begin{document}
\maketitle
{
	\hypersetup{
		hidelinks
	}
	\tableofcontents
}
\renewcommand{\arraystretch}{1.5}%Sets the vertical margin

\chapter{Managerial Economics and Theory}
Managerial economics applies microeconomics theory to business problem? For example, how to use economic analysis to make decision to achieve firm's goal to profit maximization.
\begin{itemize}
	\item \textbf{Microeconomics}\\
	      Study of behaviour of individual economic agents.
\end{itemize}
\section{Economic Cost of Resources}
\textbf{Opportunity cost} of using any resource is what the firm owners must give up to use the resourse.
\\\\
\noindent There are two kinds of resources
\begin{enumerate}
	\item \textbf{Market-supplied resources}\\
	      Resources owned by other and hired/rented/leased by the firm.
	\item \textbf{Onwer-supplied resources}\\
	      Resources owned and used by the firm.
\end{enumerate}

\section{Economic Costs}
Total economic cost is the sum of opportunity costs of both market-supplied resources and owner-supplied resources.
\begin{align*}
	Total\ economic\ cost = Explicit\ cost + Implicit\ cost
\end{align*}
\begin{enumerate}
	\item \textbf{Explicit Costs}\\
	      It is the monetary payments to owners of market-supplied resources.
	\item \textbf{Implicit Costs}\\
	      Non-monetary opportunity costs of using owner-supplied resources.
	      \begin{itemize}
		      \item \textbf{Equity Capital}\\Opportunity cost of cash provided
		      \item Opportunity cost of using land or capital owned by the firm
		      \item Opportunity cost of owner's time spent managing or working for the firm
	      \end{itemize}
\end{enumerate}

\section{Profit}
\begin{align*}
	Economic\ profit & = Total\ revenue - Total\ economic\ cost           \\
	                 & = Total\ revenue - Explicit\ cost - Implicit\ cost
\end{align*}
\begin{align*}
	Accounting\ profit & = Total\ revenue - Explicit\ cost
\end{align*}
The objective is to maximize \textbf{economic cost}.

\section{Demand}
The amount of a good or service consumers are willing and able to purchase during a given period of time is called demand.
\\\\
The \textit{demanded quantity} is represented by \(Q_d\).
\subsection{General Demand Function}
The six variables that incluence the \textit{demanded quantity} ($Q_d$) are
\begin{table}[ht]
	\centering
	\begin{tabular}{|lc|}
		\hline
		\textbf{Factor}                      & \textbf{Symbol}  \\
		\hline
		Price of good or service             & \(P\)            \\
		Income of consumers                  & \(M\)            \\
		Prices of related goods and services & \(P_R\)          \\
		Taste patterns of consumers          & \(\mathfrak{T}\) \\
		Expected future price of product     & \(P_e\)          \\
		Number of consumers in market        & \(N\)            \\
		\hline
	\end{tabular}
	\caption{Factors affecting demand function}
\end{table}
\subsubsection{General demand function}
\[
	Q_d = f(P, M, P_R, \mathfrak{T}, P_e ,N)
\]
\subsubsection{Linear demand function}
\[
	Q_d = a + bP + cM + dP_R + e\mathfrak{T} + fP_e + gN
\]
\noindent The magnitude of \(b, c, d, e, f\) and \(g\) shows the effect on \(Q_d\) on changing the respective factor.
\\
\noindent The sign of \(b, c, d, e, f\) and \(g\) shows the relationship of the factor with \(Q_d\)
\begin{itemize}
	\item Positive sign indicated direct relationship.
	\item Negative sign indicated inverse relationship.
\end{itemize}
\Cref{pattern_of_slopes} shows common dependence.
\begin{table}[ht]
	\centering
	\begin{tabular}{|c|l|c|}
		\hline
		\textbf{Variable}        & \textbf{Relation}                            & \textbf{Sign of slope parameter} \\
		\hline
		\(P\)                    & Inverse                                      & Negative                         \\
		\hline
		\multirow{2}{*}{\(M\)}   & Direct \textcolor{gray}{for normal goods}    & Positive                         \\
		\cline{2-3}
		{}                       & Inverse \textcolor{gray}{for inferior goods} & Negative                         \\
		\hline
		\multirow{2}{*}{\(P_R\)} & Direct \textcolor{gray}{for substitutes}     & Positive                         \\
		\cline{2-3}
		{}                       & Inverse \textcolor{gray}{for complements}    & Negative                         \\
		\hline
		\(\mathfrak{T}\)         & Direct                                       & Positive                         \\
		\hline
		\(P_e\)                  & Direct                                       & Positive                         \\
		\hline
		\(N\)                    & Direct                                       & Positive                         \\
		\hline
	\end{tabular}
	\caption{Commonly observed pattern of dependence}
	\label{pattern_of_slopes}
\end{table}

\subsection{Direct Demand Function}
Direct demand function shows how \(Q_d\) is related to product price \(P\) when all other variables are held constant.
\[
	Q_d = f(P)
\]
\subsubsection{Law of Demand}
When all factors other than \(P\) are constant, \(\displaystyle \frac{\Delta Q_d}{\Delta P}\) must be negative. In other words, if price increases, demand will decrease.

\subsection{Inverse Demand Function}
Traditionally, price (\(P\)) is plotted on the vertical axis and quantity demanded (\(Q_d\)) is plotted on the horizontal axis.\\
Inverse of the demand function is \textit{Inverse demand function}.
\[
	P = f(Q_d)
\]

\subsection{Graphing Demand Functions}
Demand function can help us in knowing two things:
\begin{enumerate}
	\item Maximum quantity of goods that can be purchased at a given price.
	\item Maximum price consumers would pay for the goods.
\end{enumerate}

\bibliographystyle{unsrt}
\bibliography{main}
\end{document}